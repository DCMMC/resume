%%
%% Copyright (c) 2018-2019 Weitian LI <wt@liwt.net>
%% CC BY 4.0 License
%%
%% Created: 2018-04-11
%%

% Chinese version
\documentclass[zh]{resume}

% File information shown at the footer of the last page
\fileinfo{%
  \faCopyright{} 2019 Wentao Xiao
  %\creativecommons{by}{4.0},
  %\githublink{liweitianux}{resume},
  %\faEdit{} \today
}

\name{文韬}{肖}

% \keywords{Deep Learning, Keras(Tensorflow), Arch Linux, Machine Learning, Full Stack}

% \tagline{\icon{\faBinoculars}} <position-to-look-for>}
% \tagline{<current-position>}

% \photo{<height>}{<filename>}

\profile{
  \mobile{177-7511-0118}
  \email{wentao.xiaoo@mail.dhu.edu.cn}
  \github{DCMMC} \\
  \university{东华大学(“双一流”,211)}
  \degree{计算机科学与技术 \textbullet 本科}
}

\begin{document}
\makeheader

%======================================================================
% Summary & Objectives
%======================================================================
{\onehalfspacing\hspace{2em}%
熟悉前后端分离的开发方式,前端领域有 Vue,mpvue 开发经验,后端熟悉 Django 开发和 MongoDB 数据库,接触过 Qt,Android 和 JavaFX 等原生应用开发,在 Docker 运维上有比较丰富的经验,同时也有小程序开发经验。
善于在自我驱动(self-motivated)下学习陌生领域的知识,有较为丰富的工程经验,对计算机和深度学习有关技术有浓厚的兴趣。
有 4 年的 Linux 长期使用经验,熟悉 JavaScript、Java、Python、C++ 等语言编程。
积极实践自由开源精神,
在 \link{https://github.com/DCMMC}{GitHub} 上分享多个开源项目,
并积极参与其他多个开源项目。
\par}

%======================================================================
\sectionTitle{学习经历与成绩}{\faGraduationCap}
%======================================================================
\begin{educations}
	\education%
	{2016.09}%
	{东华大学}%
	{计算机科学与技术学院}%
	{计算机科学与技术}%
	{前六学期学分绩点:3.81,排名:5/62}%
\end{educations}
\begin{itemize}
	\item 奖项:天骥奖学金,学习优秀奖,优秀社会工作奖等%
\end{itemize}

%======================================================================
\sectionTitle{项目经历}{\faCode}
%======================================================================
\begin{itemize}
	\item \link{https://github.com/DCMMC/mp-vote}{\texttt{mp-vote}}: 使用 vue,mpvue 和 vant-webapp 开发的一个简单的投票小程序。
	\item \link{https://github.com/DCMMC/ParkingLot}{\texttt{ParkingLot}}:
	智能停车场管理系统,前端使用 Vue, ElementUI, Vuetify, Three.js, 后端使用 Django, Django-Channel, Celery, 数据库使用 MongoDB, MongoEngine, 车牌识别使用基于 Keras 和 Tensorflow 的 HyperLPR。使用 Qt(C++)编写了一个简单的停车场建模软件(导出为 json 数据)。整个项目使用 Docker-Compose 搭建环境和配置。
	\item \link{https://github.com/DCMMC/FunCamera}{\texttt{FunCamera}}: 使用当时刚推出的 tensorflow-lite 编写的安卓相机应用,使用 tensorflow-lite 运行实时物体检测的深度学习模型,拍照后可以对照片进行编辑和基于深度学习的油画风格化处理。
	\item \link{https://github.com/DCMMC/DHUCourseSelector}{\texttt{DHUCourseSelector}}: 使用 Java 以及 JavaFX 作为图形库编写的跨平台可视化选课软件。
	\item \link{https://github.com/DCMMC/SimpleReg2Automata}{\texttt{SimpleReg2Automata}}: 使用 Java,graphviz 以及 JavaFX 作为图形库编写的展示正则表达式 $\Rightarrow \epsilon$-\texttt{NFA} $\Rightarrow \texttt{DFA} \Rightarrow$ \texttt{minimized DFA} 的转化过程的可视化界面。
	\item 运维工作: 在学校为材料学院人事招聘系统负责运维,使用 Docker-Compose 部署其系统。
\end{itemize}

%======================================================================
\sectionTitle{实习经历}{\faBriefcase}
%======================================================================
\begin{experiences}
	\experience%
	[2018.9]%
	{2018.12}%
	{后端工程师 @ 上海方锤智能科技有限公司(初创公司)}%
	[\begin{itemize}
		\item 在树莓派上使用 Django-Channel 搭建异步 WebSocket 连接,用于教练机与多台学员机器之间的实时控制与传感器数据实时传输
		\item 使用 MongoDB, Vue 和 Vuetify 搭建的一个学员信息录入系统
		\item 配置 Docker-Compose 简化部署
	\end{itemize}]
	\separator{0.5ex}
	\experience%
	[2019.7]%
	{2019.9}%
	{实习生 @ 鹏城实验室网络安全中心}%
	[\begin{itemize}
		\item 研究课题为使用深度学习有关技术进行勒索病毒分类和网络流量分类
		\item 使用鹏城实验室的超算平台运行实验
	\end{itemize}]
\end{experiences}

%======================================================================
\sectionTitle{技能和语言}{\faWrench}
%======================================================================
\begin{competences}
	\comptence{编程}{%
		Python, Java, ECMAScript(JavaScript), C++, Matlab, C, Bash Shell
	}
	\comptence{工具}{%
		Git, CMake, Vim, \LaTeX\ , Markdown
	}
	\comptence{数据分析}{%
		Matplotlib, Keras, Tensorflow, Scikit-learn
	}
	\comptence{全栈开发}{%
		Django, Node.js, Vue, WebSocket, Redis, MongoDB, Oracle DB, Docker
	}
	\comptence{原生应用开发} {
		JavaFX,Qt(C++/Python),Android
	}
	\comptence{\icon{\faLanguage} 语言}{
		\textbf{英语} CET4,CET6
	}
\end{competences}

\end{document}
